\documentclass{article}
\usepackage[utf8]{inputenc}
\usepackage[american]{babel}
\usepackage{csquotes}
\usepackage{pdfpages}
\usepackage{hyperref}
\usepackage{color,soul}
\usepackage[backend=biber,style=numeric,hyperref=true,natbib=true,autocite=plain,sorting=none]{biblatex}
\usepackage[margin=1in]{geometry}
\usepackage{rotating}

% https://tex.stackexchange.com/a/223163
\usepackage{silence}
\WarningFilter{latex}{Text page}

% paragraph spacing - see https://www.overleaf.com/learn/latex/Paragraph_formatting#Line_spacing
\usepackage{setspace}
\setlength{\parindent}{4em}
\setlength{\parskip}{1em}
\renewcommand{\baselinestretch}{1.5}

% this package and the below text is to force images to be added to the given section and subsection. See https://tex.stackexchange.com/a/235312 for more information
\usepackage{placeins}
\let\Oldsection\section
\renewcommand{\section}{\FloatBarrier\Oldsection}

\let\Oldsubsection\subsection
\renewcommand{\subsection}{\FloatBarrier\Oldsubsection}

\let\Oldsubsubsection\subsubsection
\renewcommand{\subsubsection}{\FloatBarrier\Oldsubsubsection}

\addbibresource{references.bib}

\newcommand{\maintitle}{NASA SUITS Proposal: Lunar Locator}
\newcommand{\schoolname}{Stevens Institute of Technology}

\title{
  \maintitle \\
	\large \schoolname}

\date{October 2019}
\author{Joshua Schmidt, Emily Weeks, Dylan Regan, John Cinquegrana}

\usepackage{graphicx}

\begin{document}

\maketitle

\bigskip
\bigskip
\bigskip
\bigskip

\begin{figure}[!htb]
  \centering
  \includegraphics[width=0.5\textwidth]{assets/logo.png}
  \label{fig:logo}
\end{figure}

\newpage

\newcommand{\contactsection}{}

\includepdf[scale=1.0, offset=0 0, pages=1, pagecommand=\contactsection]{assets/contactinfo.pdf}

\iffalse

\begin{center}

\bigskip
\bigskip

{\setstretch{0.8}

\textbf{\schoolname}

\href{http://maps.google.com/?q=1+Castle+Point+Terrace,+Hoboken,+NJ,+07030}{1 Castle Point Terrace, Hoboken, NJ, 07030}

\bigskip
\bigskip

\textbf{Team Contact}

Joshua Schmidt

\href{mailto:jschmid3@stevens.edu}{\nolinkurl{jschmid3@stevens.edu}}

\href{tel:19085317087}{+1 (908) 531-7087}

\bigskip
\bigskip

\textbf{Team Members}

Joshua Schmidt -- Team Lead and Electrical Design

\href{mailto:jschmid3@stevens.edu}{\nolinkurl{jschmid3@stevens.edu}} -- Junior/Computer Engineering

\bigskip
\bigskip

John Cinquegrana -- Software Designer 

\href{mailto:jcinqueg@stevens.edu}{\nolinkurl{jcinqueg@stevens.edu}} -- Sophomore/Computer Science

\bigskip
\bigskip

Emily Weeks -- Project Management

\href{mailto:eweeks@stevens.edu}{\nolinkurl{eweeks@stevens.edu}} -- Junior/ Engineering Management

\bigskip
\bigskip

Dylan Regan -- Software Development

\href{mailto:dregan@stevens.edu}{\nolinkurl{dregan@stevens.edu}} -- Sophomore/Computer Science

\bigskip
\bigskip

Ronald Anker -- Electrical Design

\href{mailto:ranker@stevens.edu}{\nolinkurl{ranker@stevens.edu}} -- Graduate Student/Electrical Engineering

\bigskip
\bigskip

\textbf{Faculty Advisor}

Mukundan Iyengar

\href{mailto:miyengar@stevens.edu}{\nolinkurl{miyengar@stevens.edu}}

\href{tel:12012165603}{+1 (201) 216-5603}

\bigskip
\bigskip
\bigskip
\bigskip
\bigskip
\bigskip
\bigskip
\bigskip
\bigskip
\bigskip
\bigskip

\begin{tabular}{@{}p{.5in}p{3.5in}@{}}
Approved: & \hrulefill \\
& Iyengar, Mukundan, 10/25/2019\\
\end{tabular}
}

\end{center}

\fi

\newpage

{\setstretch{1.0}
\tableofcontents
}

\newpage

\section{Letter of Intent}

\begin{center}

\bigskip
\bigskip
\textbf{Institution}

\schoolname

\bigskip

\textbf{Team Members}

\bigskip
\bigskip

\begin{tabular}{||c c c c||} 
 \hline
 Name & Email & Academic Year & Major \\ [0.5ex] 
 \hline\hline
 Joshua Schmidt & \href{mailto:jschmid3@stevens.edu}{\nolinkurl{jschmid3@stevens.edu}} & Junior & Computer Engineering \\ 
 \hline
 Dylan Regan & \href{mailto:dregan@stevens.edu}{\nolinkurl{dregan@stevens.edu}} & Sophomore & Computer Science \\
 \hline
 Emily Weeks & \href{mailto:jschmid3@stevens.edu}{\nolinkurl{eweeks@stevens.edu}} & Junior & Engineering Management\\
 \hline
 John Cinquegrana & \href{mailto:jcinqueg@stevens.edu}{\nolinkurl{jcinqueg@stevens.edu}} & Sophomore & Computer Science \\
 \hline
 Ronald Anker & \href{mailto:ranker@stevens.edu}{\nolinkurl{ranker@stevens.edu}} & Graduate Student & Electrical Engineering \\
 \hline
\end{tabular}

\bigskip
\bigskip

\textbf{Advisor}
\bigskip
Mukundan Iyengar

\href{mailto:miyengar@stevens.edu}{\nolinkurl{miyengar@stevens.edu}}

\href{tel:12012165603}{+1 (201) 216-5603}

\bigskip
\bigskip
\textit{We plan to submit a proposal for the NASA SUITS Challenge.}

\end{center}

\newpage

\section{Technical Section}

\subsection{Abstract}

The goal of this project is to use innovative strategies and resources to aid astronauts in Extravehicular Activities (EVAs) as they navigate and complete tasks outside of the lunar landing module. To accomplish this, we will be using emerging technologies including Augmented Reality and network-connected sensors, in order to create intuitive displays of vitals and tasks, show navigation details, and allow astronauts to see in low visibility areas.

By employing a Microsoft Hololens \autocite{mrbasics} \autocite{mannedspaceflight} and a large data acquisition system, we plan to implement image processing and localization tools in order to augment an astronaut's environment, making it easier for the astronaut to complete given tasks. Our implementation includes 9 key functions, detailed in the next section:

\begin{enumerate}
\item SLAM and Radiowave Tracking
\item Eye Tracking Illumination
\item Suit to Rover to Ground Control Interaction
\item The HUD and User-Interface
\item Task-List Presentation
\item Suit Control API
\item Button Board and Task-List Interaction
\item Return to Home Protocol
\item Caution and Warning System
\end{enumerate}

\subsection{Design Description}

\subsubsection{Localization}

The task of localizing a space suit on the lunar surface is relatively complicated as there is no GPS or cell towers, and very little infrastructure in general. All of the localization needs to be done on the lander or the space suit itself to avoid the latency of sending data to Earth and back. To accomplish this, we will be employing several forms of tracking to localize the space suit during an EVA, including SLAM and ultra wide-band radio tracking. By filtering these two tracking algorithms, we will be able to accurately localize the suit to within a meter at any given time.

\subsubsection{SLAM Tracking}

SLAM --- Simultaneous Location and Mapping\autocite{slam_riisgaard}--- is ideal for localization in changing environments that are difficult to map. The lunar surface contains many intricate features that would be very challenging to add to a comprehensive map, and it is constantly changing as astronauts perform experiments or move equipment; making it the perfect situation to use SLAM. There are several key aspects to the SLAM algorithm - landmark extraction, data association, state estimation, as well as state update and landmark update \autocite{slam_riisgaard}. The data to feed into the algorithm can come from any sensor but we will be using RGBD camera data (which includes depth information). In the initial state, the location is set at zero with a high level of confidence\autocite{rgbtocad}. The odometry change (estimated position) is zero because it did not move since the initial observation, and this is fed into an EKF --- Extended Kalman Filter. The camera data is then processed to extract landmarks and associate the data with previous landmarks seen. Landmarks that have not been seen are added to the EKF as new observations so they can be re-observed in the future. New data is then fed into the EKF for a re-observation, and this estimated odometry is used to restart the cycle. The data flow can be seen in Figure \ref{fig:slamdiagram} in the Appendix.

Landmark extraction and constant state estimation are the main features of SLAM. Drift will occur as the astronaut continues to move from the initial position, and the amount of error is directly dependent on the quality of measurements from the sensors, in addition to the way landmarks are extracted. In order to improve localization further, we will be using Ultra-Wideband Radio Tracking in addition to SLAM, potentially feeding this data into the EKF for greater accuracy.

\subsubsection{Ultra-Wideband Radiowave Tracking}

Ultra-Wideband radios are radios that have bandwidths larger than 20\%, or absolute bandwidths of more than 500 MHz \autocite{localization}. This large bandwidth improves reliability, as the signal has many different frequency components, and this increases the probability that the signal can travel around obstacles. The main advantage of using UWB radios for localization is its high ranging and positioning accuracy.

UWB radios can have as high as 1cm accuracy for determining the location of an object \autocite{localization}. And adding more UWB radios to the network only increases its accuracy. There are three different types of positioning systems for UWB - time-of-arrival, direction-of-arrival, and signal-strength. Angle-of-arrival based positioning techniques measure the angle of the target node seen at reference nodes, which is accomplished using antenna arrays. This approach is not suitable for UWB positioning because antenna arrays increase system cost, and the number of paths can be very large because of the size of the band used.

A much better approach for UWB is signal strength, where the distance between two nodes is calculated based on the energy of the received signal at one node. This technique requires at least three reference points to find the 2-D location of a given node, or four to find the 3-D location, as shown in Figure \ref{fig:uwbdiagram} in the Appendix. To determine the distance from signal strength measurements, the characteristics of the channel need to be known well. The Cramer-Rao lower bound (CRLB) for distance estimate \textit{d} is defined by: $ \sqrt{V_{ar}(d)} >= \frac{ln(10)}{10} \cdot \frac{\sigma _{sh}}{n _{p}} \cdot d $, where $n _{p}$ is the path loss factor and $\sigma _{sh}$ is the standard deviation of the log-normal channel shadowing effect \autocite{localization}. This equation shows the minimum achievable variance of distance, or the theoretical limit of accuracy for the signal strength approach. Signal strength measurements can be used in conjunction with time delay measurements of other nodes, which can improve localization accuracy \autocite{uwbindoorpositioning}.

Time-based positioning relies on measurements of travel times of signals between nodes \autocite{localization}. If the two nodes have a common clock, the receiving node can get the time of arrival and use a time-stamp from the signal to determine the delta. The best achievable accuracy of a distance estimate \textit{d} is defined by: $ \sqrt{V_{ar} (d)} >= \frac{c}{2 \cdot \sqrt{2} \cdot \pi \cdot \sqrt{SNR} \cdot \beta} $, where $c$ is the speed of light, $SNR$ is the signal-to-noise ratio, and $\beta$ is the effective signal bandwidth. Unlike signal-strength techniques, the accuracy of a time-based approach can be improved by increasing the $SNR$ or the signal bandwidth. Because UWB signals have very high bandwidths, it allows for very accurate location estimates. For example, if a UWB radio receives a 1.5 GHz bandwidth pulse, the accuracy is calculated with 1 cm with SNR = 0 db.

Time-based schemes provide very good accuracy due to the high time resolution (large bandwidth) of UWB signals. They are less costly than AOA-based processes, and has better quality range information than signal strength measurements. Therefore, we will be using a time-based or possibly a signal-strength time-of-arrival combination algorithm to get the location of the space suit.

Physically, there will be multiple radios placed strategically on the lunar surface, acting as base points or nodes. There will be an additional UWB radio embedded in the suit, so that the time of arrival can be calculated at the spacesuit node (see Figure \ref{fig:uwbdiagram} in the Appendix). NTP --- Network Time Protocol --- which is ubiquitous in computer network systems, will be used to sync the clocks for each of the radios.

There was also the idea of using LoRa radios in a mesh network for localizing the space suit instead of the UWB radios \autocite{lorapositioning}. LoRa would have allowed for lower-cost positioning, but at the cost of far less accurate localization data. With complex software and a large amount of LoRa radios, these drawbacks can be mitigated, but for the purposes of this experiment we decided to stick with UWB.

\subsubsection{Eye Tracking Illumination}

The lunar surface has widely varying degrees of illumination. This poses an issue for astronauts as they perform their EVAs, collecting samples, performing experiments and doing maintenance. There are various ways to approach this problem. With no extra hardware required, computer vision and image processing could be used to artificially increase the contrast and highlight dark areas in an image. However, this requires relatively high computational power for accurate results, and a realistic blend of virtual and augmented reality.

For better results, we decided to use an LED array on the suit to illuminate dark areas of the astronaut's vision. Initially we thought of using a single light source on a gimbal, which moves in the direction the astronaut is looking at a given time using eye tracking. Since this requires moving mechanical parts which are more prone to failure, we decided the better solution is to have fixed lights on the suit in different locations, turning on when needed. These lights should illuminate only dark areas of where the astronaut is looking at a given time in order to preserve power. 

To summarize, our proposed solution consists of various light sources positioned strategically on the space suit, with differing focal lengths and brightness levels. Computer vision and image filtering is used to determine the dark areas of the astronaut's field of view, and tracking technology shown in Figure \ref{fig:hololenseyetracking} is used to deduce in what direction the astronaut is looking. Additionally, the localization data is used to determine exactly where the astronaut is looking relative to the lunar surface. With this information, our system will then intelligently light the astronaut's surroundings.

\subsubsection{Suit to Rover to Ground Control interaction}

High speed internet is something that is quite sparse on the lunar surface (coming soon \autocite{mooninternet}). Relaying information between the astronaut, the devices on the astronaut's suit, and the command module is essential to any EVA. In our solution, information will be flowing through two different highways from the suit to a router representing the lunar rover or lunar lander. The information flow is handled completely on the backend with no interaction needed by the astronaut.

The primary information highway will be controlled by a Raspberry Pi low-power computer located on the space suit itself (see Figure \ref{fig:networkdiagram}). It will be running Robot Operating System (ROS) that exposes topics to all the nodes connected to the network, including the different sensors on the suit and the "cloud" --- the server on the lunar rover. Due to limitations of ROS, this highway is unable to handle the bandwidth necessary for camera streaming without significant delay; thus the creation of the secondary direct connection to the server \autocite{dataperformance}. The ROS server in the pi will handle traffic from the IMU (Inertial Measurement Unit), UWB radio system, button board, and all other low-bandwidth sensory inputs. The benefit of this system is that even if connection is lost to the cloud, the sensors will still give feedback to the HUD and everything on the suit will still work. The only functionality lost would be the compute power from the cloud, i.e. the real-time tracking and camera vision.

The secondary information highway will be a direct upload to the server. All camera tracking will be uploaded to the cloud through this highway. It is unique from the other highway primarily because of it's ability to handle larger data packages at a higher speed.

For test purposes, the ideal network will be simulated using two wifi routers acting in bridge-mode --- one "travel router" on the space-suit itself, connecting all of the sensor nodes and the raspberry pi; and one long-range router on the lunar rover. The cloud compute power will be simulated on a laptop on the rover side.

\subsubsection{The HUD and User-Interface}

Our goal for the User-Interface (UI) is to be as minimal and intuitive as possible. It is very important for the UI to not obstruct the astronaut's visibility with extraneous information, as we do not want to hinder or slow down the astronauts as they complete tasks, with complicated or unintuitive controls. In order to accomplish this we will be using a button board mounted on the suit. We decided to use this over other methods of inputs because of its simplicity. A button board keeps false positive inputs to a minimum, which making it easy to interact with the UI.

The user-interface will have a main page showing diagnostic data from the space suit, as depicted in Figure \ref{fig:suitdiagnostics}. This page will display any information the astronaut needs to know at all times, and can be customized by the astronaut prior to the EVA. There will also be a list of tasks, detailed in the next section, which would be on the next UI page. Switching between these two interfaces will be just a simple button press, so it can be done quickly.

The last page we want to add for the astronaut is a notes page. This will be useful especially for longer or more complex tasks, allowing for astronauts to read notes they wrote preparing for the EVA. Important warnings, such as if oxygen levels drop below a specified point, will appear regardless of which interface the astronaut is using, in order to ensure no information is lost if the astronaut is not viewing the main diagnostics page.

\subsubsection{Task-List Presentation}

With each EVA comes a specific set of goals and tasks, which changes dynamically based on the mission. Therefore, we plan to have the Heads Up Display (HUD) show each goal the astronaut is to achieve in a concise and specific way, getting this data from a stateful, dynamic api.

As a current solution to showing these task-lists, NASA employs an arm folder of printed instructions. As we can see in Figure \ref{fig:spacesuitdisplay} this requires the astronaut to divert their attention from their current task and, if need be, physically flip through the arm folder.

A HUD display for showing the tasks in a concise and non-intrusive way would be a great improvement over the current analog method. The astronaut would see a drop-down menu of current tasks, with a recursive structure to allow for sub-tasks and sub-sub tasks as part of larger, main tasks (see Figure \ref{fig:tasklistdropdown}). This allows for very fine detail to be displayed to the astronaut without being overly distracting. Photos and videos can also be added to the task list to allow for illustrating the task graphically.

\subsubsection{Suit Control api}

The tasks will be sent to the suit through a unified graphql api. Graphql is a new http / REST framework that allows for data to be queried in a standard way. By using graphql, we can reduce utilized bandwidth and only send the data that is needed at a given time. So if there is a large image file or diagram that needs to be sent to illustrate the second task, it will not be queried until the first task is completed.

This unified api will get data from a ground-control web-application, allowing authenticated users to add new tasks for the astronaut to complete. Additionally, the astronaut can add notes for individual tasks as they prep for the EVA, with attachments for media like pictures or video. The api will interface between the admin web app (shown in Figures \ref{fig:groundcontrolsuitmetrics} and \ref{fig:taskmanagment}), the database and the hololens itself, also ingesting data from ROS topics to save to the database. This data collection allows for easy access to historical data through the admin panel, so ground control can see sensor data timestamped through the duration of the EVA.

\subsubsection{Button Board and Task-List Interaction}

Last year we used a glove to interact with the button board. The glove contained spring gauges that were used to interpret different hand gestures. While this was a good idea in concept, in practice there were often accidental gestures, and the calibration process took time.

Instead of using spring gauges, we decided to go back to the basics and use a button board for navigating through the HUD interface. The button board includes navigation buttons to change views and select different options. It will also include programmable buttons that the astronaut can quickly change via the web gui for different missions, such as a shortcut to navigate back home or turning the lighting system to max brightness.

On the task list screen, the astronaut will be presented with a drop-down list of tasks that they will need to complete in order. An example of this is shown in Figure \ref{fig:tasklistdropdown}. The tasks will be rendered in a hierarchy, with tasks and sub-tasks and sub-sub-tasks. Once the astronaut is done with a given task, they can switch to the next with the navigation buttons. This process will be easy and intuitive for the astronaut, with clearly labeled buttons.

\subsubsection{Return to Home}

The “Return to Home” Figure \ref{fig:wayhome} message will appear only in emergency situations. The system will continuously calculate the astronaut's position relative to the spacecraft access point. In times of emergency when an astronaut must get inside the spacecraft as quick as possible, this panel will appear and display a red indicator pointing toward the access point: an arrow when the access point is in the field of view and a red crescent when not in the field of view.

\subsubsection{Caution and Warning System}

The caution and warning system will continuously monitor various metrics and employ both audio and HUD notifications to convey the issues. The system will have two sets of thresholds for each metric being monitored: one for caution and one for warning. The metrics that the system will monitor are as follows:  all suit telemetry, all inertial data, location data and task addition and deletion. For the suit telemetry, all cautions will be displayed in yellow on the panel. A caution will not trigger the panel to appear, but audio will notify the astronaut of the caution. A warning will be displayed in red on the panel and will trigger the panel to appear automatically. The panel will also be accompanied by audio indicating the warning. For example, if primary or secondary $O_{2}$ drops below a dangerous pressure, the Suit Status panel will appear and the $O_{2}$ bar graphic will be displayed in red. The inertial data will follow similar operation: a caution will trigger audio only and a warning will trigger the panel to appear accompanied by audio. Thresholds for the inertial data will be set such that a warning occurs only if the astronaut seems to be out of control. The appearance of this panel will hopefully aid the astronaut in regaining control as he or she can now see which degrees of motion are unstable or too high. For the task manager, the addition or deletion of a task can be set to caution or warning at the discretion of the operator editing the tasks. A caution will be indicated with audio and a warning will cause the Task Manager panel to appear, displaying the updated main task and subtask-expansion where an addition or deletion has occurred.

\subsection{Concept of Operations (CONOPS)}

The goal of this project is to create a user-friendly interface to help facilitate tasks on EVA walks. Using augmented reality, we want to alleviate the amount of procedures that astronauts have to remember to accomplish the mission. In practice we hope that astronauts will be able to rely on this system to navigate so they can concentrate more on the overall goals of their missions and less on trivial tasks. Accomplishing this through accurate localization and unified sensor network, we aim to provide this information with low latency compared to other more complex and time consuming methods.

With the advent of a display for suit telemetry, we aim to help put astronauts’ minds at ease, as they will have full access to, and are always aware of, the conditions they are in. We hope they utilize this interface as their main source of information, and can navigate it in an intuitive way. Using a form to create mock data in real time is the best way to test this capability.

The task list is one of the biggest advantages of moving to an augmented reality interface. With a heads up display (HUD) of current and planned tasks, we hope to provide astronauts with an easy to use and accessible hub to allow for a more natural, non-restrictive, way to get information. This implementation, as stated before, is designed to handle dynamic changes which gives the astronaut the opportunity to adjust tasks mid-mission if problems arise. This feature achieves the goal of aiding astronauts in their tasks and helps reduce the stress in memorizing all the nuances of the mission.

The way back, inertial monitor, localization, and warning systems are our solution for further making the space suit a command center for astronauts. It helps aid the astronaut in navigating dangerous situations, improving efficiency and facilitating communication.

\subsection{Human-in-the-loop (HITL) Testing}

\begin{enumerate}

\item Eye Tracking Illumination: Our illumination protocol will be tested against a randomly chosen group of students. The students will be put in a dark environment and asked to perform various physical tasks such as assembling basic puzzles and tightening screws. The students will perform the tasks in three groups, one with the eye tracking illumination, one with a stationary head-mounted flashlight, and the last with a hand-held flashlight.

\item Caution and Warning System: The second test will take the same group of students, and have them perform menial tasks while the suit diagnostics show on their HUD. The students will be asked to describe what information they feel may be helpful, what displays too obtrusive to the task, and what information may be missing.

\item HUD Interaction: The third test will have the students performing routine maintenance on a "lunar rover", simulated with a simple RC car. While the student is encouraged to make use of the button board, the test focuses on the clarity of the HUD, so they have the option to interact only with one of our team members who will advance the HUD for them.

\item Button Board: The fourth test will see the students performing tasks as they appear on the Task-List protocol in their HUD. The students will be wearing ski-gloves to simulate limited hand movement and feeling. They will be preforming tasks similar to the first test, in the setting of a simulated dig site. They will need to interact with the button board in order to move forward in the task; for the purpose of this test the button board and HUD will be their only source of information. The test will be evaluated on how quickly and effectively the students can finish the task, as well as how comfortable they are under the conditions.

\item Localization Trials: The localization of the astronaut on the lunar surface is the cornerstone of our application. As such, robust testing is necessary. As we have access to only one Hololens, and one set of sensors, bulk trials will be impossible. The effectiveness of our SLAM tracking will be put to a weekly benchmark. When more rigorous testing is called for, additional student volunteers will be asked to help. They will be tasked with navigating through environments in various degrees of illumination with varying amounts of obstacles. Their qualitative description of how helpful our software was, as well as our quantitative statistics, will measure the robustness of our application.

\end{enumerate}

\newpage

\subsection{Project Schedule}

%\begin{sidewaysfigure}
\includegraphics[width=0.95\textwidth, keepaspectratio]{assets/ganttscreenshot1.png}
%\end{sidewaysfigure}

\newpage

\subsection{Technical References}

\printbibliography

\newpage

\section{Public Outreach}

Our plan with outreach is to spread the word about the opportunities that NASA provides, such as competitions like this as well as potential careers and opportunities at NASA. Moreover, we plan on emphasizing the opportunities that STEM offers and how students can get exposure early. We plan on targeting high school and middle school students who have interest in STEM or space and aeronautics specifically.

There are three groups we currently plan on visiting; Bayonne High School in Bayonne, NJ, Hoboken High School in Hoboken, NJ and Raritan High School in Hazlet, NJ. As of now, we have two presentations confirmed at Raritan High School (letter in Appendix) and Bayonne High School, and discussions with Hoboken High School are ongoing.

Our presentation will follow a straight-forward layout, uniquely created by our team. The presentation will follow the below outline:

\begin{enumerate}
\item General Introduction What does STEM have to offer?
\item What NASA provides to the world
\item What NASA can provide to you
\item What you can provide to NASA
\item Personal STEM/NASA Experiences
\begin{enumerate}
\item Emily Weeks
\item Joshua Schmidt
\item Dylan Regan
\item John Cinquegrana
\item Swag (from JSC Education Office and Public Affairs Office) and Questions
\end{enumerate}
\end{enumerate}

While each presentation will be slightly different, the general theme and key points are stated in the section below, accompanied by pictures we will present to the students.

\subsection{General Introduction: What does STEM have to offer?}

Science, Technology, Engineering, and Mathematics, colloquially known as STEM, offers an unobstructed view into the universe. It provides a powerful method used to understand and see the world around us, and better yet, manipulate it. STEM is what allows us to see thousands of galaxies billions of miles away and thousands of atoms that make up the head of a nail. It is what allowed humanity to first sail between the continents (and find our way back), to first take flight, as well as explore the cosmos and the secrets it holds. It allows us to cure diseases that plagued the human race for centuries and give mobility back to those who have lost it. Overwhelmingly  though, STEM gives each and every one of us the power to change the world for the better.

\subsection{What NASA provides to the world}

\begin{figure}[!htb]
  \centering
  \includegraphics[width=0.75\textwidth]{assets/smilingatthemoon.jpg}
  \caption{Moon Shot}
  \label{fig:smilingatthemoon}
\end{figure}

For the last 60 years, above scientific advancement and community, NASA has provided inspiration to the world. Shortly after its creation, NASA was thrown into the kiln with one of the hardest problems humanity had ever faced: putting man on the moon. With the time, sweat, and dedication of over 50,000 Americans over the course of 7 years, NASA brought humanity to moon and with it ushered in a new era of human progression: transforming humans into a multi-planetary species. The accomplishment of such a incredible feat was so influential that now any task seemingly-impossible or any goal too lofty is often said to be a “moon shot”. That event and the work required to achieve it left an indelible mark on society and has been inspiring dreamers in every generation since, ourselves included.

Furthermore, NASA’s mission to benefit humanity, was quiet literally illuminated in its motto: “For the benefit of all”. With this mentality, hundreds of thousands of diligent employees have been working over the last 60 years to advance society in areas including, but most not limited to, space, aeronautics, medicine, biology and the environment. NASA tackles the hardest problems of our time and moreover, outsources all of its discoveries. The work done, research conducted, discoveries made, are all done in an effort to advance humanity.

\subsection{What NASA can provide to you}

Working with NASA can provide great experiences with some of the hardest problems on and beyond this world, and incredible applications in wide world STEM. NASA provides ample opportunities such as the NASA Suits Challenge, and University Student Design Challenge at NASA’s Glenn Research Center for undergraduate students, or The Great Water Design Challenge for high school students and even Explore Earth: NASA Earth Science Missions for educators. All of these programs provide valuable information to students and educators alike. These opportunities are life changing on many fronts and are a great stepping stone to a career at NASA. 

\begin{figure}[!htb]
  \centering
  \includegraphics[width=0.75\textwidth]{assets/2019team.jpg}
  \caption{Stevens NASA Suits Challenge 2019 Team}
  \label{fig:2019Team}
\end{figure}


\subsection{What you can provide to NASA}

During this part of our presentation we will go through our NASA Suits Project, its results, and our experience working on a NASA project while still in college. We plan to have videos and pictures of a look forward into space and aeronautics in the next 30 years to show to the students.

\subsection{Personal STEM / NASA Experiences}

John Cinquegrana: Computers and logistics have been my ambition for as long as I can remember. I wrote my first program at the age of 7, and have been obsessed with logical problems and puzzles ever since. My first NASA experience was competing in the 2017 STEMnauts competition. Since the beginning of my college career one year ago, I've had the opportunity to teach a physics recitation, win third place at the national Altice Innovation Roadshow Hackathon, and develop a web-app under the SalesForce cloud. All of these experiences gave me amazing memories, and even greater opportunities. I'm hoping to give my experiences a test while engaging in this NASA SUITS competition. Hopefully my team and I can create some amazing applications.

Joshua Schmidt: I have been involved in science and engineering for nearly my entire life. Since middle school in the gifted and talented program I have been programming and working with computer hardware, experimenting with OCR, networking, ML, and emerging technologies on various platforms. In high school I was captain of my First robotics team, working closely with the design and programming divisions to create amazing robots. Now in Stevens I am studying Computer Engineering with a minor in Computer Science. I do freelance work building web and native apps, and this past summer worked for Nokia Bell Labs developing their autonomous robot platform. I completed this challenge last year and went to Houston to present, and I had an amazing time talking with people who could be using our technology in the future. There was so much that we accomplished and even more that we could improve, which is why I am back again to take another shot at this challenge.

Emily Weeks: As an Engineering Management student at Stevens Institute of Technology, I have had many opportunities in the STEM community. I have visited Kennedy Space Center in Cape Canaveral Florida where I observed a launch and participated in a tour. I have also visited the Smithsonian National Air \& Space Museum. Both experiences have opened my eyes to space exploration and the impact NASA has on our everyday life. From those experiences I began to dream of being an astronaut, which led me down the path of STEM throughout elementary and high school; eventually to Stevens Institute of Technology where I am now participating in this challenge.

Dylan Regan: Ever since I was little, I've wanted to know how the world works. As a kid I spent hours just testing how water went up and down as I got in and out of the tub. I would keep finding something, and wondering about it until I got my answer. But space is a concept I never got my answer for. No matter how many Neil Degrasse Tyson videos I watched, I couldn't find the satisfaction I was looking for. I just kept having more questions. During a family visit to both The Kennedy Space Center and The Boeing Factory, I got to see just how much others share that same wonder of the world; I saw what they were doing and wanted to help as well. I am now currently studying computer science at Stevens Institute of Technology. I love any kind of puzzle or problem solving question. And while doing these problems, I often find a way that many others would overlook or miss at first. I am now hoping to use my unique way of thinking to tackle the NASA Suits Challenge.

Ronnie Ankner: Space has always had a special place in my heart. I grew up always curious, always wanting to know more, and space held all of the curiosities of the universe, a seemly bottomless cauldron of discovery. NASA in turn also had this aire of discovery and something extraordinary to me growing up and like many, I had dreams of being an astronaut. STEM naturally gave me a perfect conduit to fulfill my curiosities and understand the world around me. I didn’t realize I wanted to do engineering until late in high school during an ordinary bottle rocket project that I took way over the top: building my own test stand at home to test different design iterations, studying fluid dynamics and partial differential equations when I barely had a grasp on a regular derivative, and using excel to make a rudimentary state calculator to give all of the important metrics of the rocket at tiny steps in time.

Fortunately, I have had the opportunity to explore amazing opportunities that STEM has to offer, both at my college and in two NASA internships. My first internship at Glenn Research Center in Ohio, gave me the opportunity to work in aeronautics on a team of other interns in different majors and from schools across the country. My second internship at Johnson Space Center in Houston Texas opened me up to everything space had to offer. Within my first week there, I had written software that will be going into space shortly. And, during my time, I worked with various NASA employees and a PhD at a local college to create a totally experimental device that I was the point contact for. I got a chance to realize my dreams of being an astronaut (only for a day though), doing a full day of training and lab visits, and meeting some of the smartest most accomplished people I have ever met. But, above all else, I found a home there and incredible sense of family both with my fellow interns and work colleagues.

\begin{figure}[!htb]
  \centering
  \includegraphics[width=0.75\textwidth]{assets/staringatrover.jpg}
  \caption{Day-in-the-life of an Astronaut 1}
  \label{fig:staringatrover}
\end{figure}

\begin{figure}[!htb]
  \centering
  \includegraphics[width=0.75\textwidth]{assets/vrtest.png}
  \caption{Day-in-the-life of an Astronaut 2}
  \label{fig:vrtest}
\end{figure}

\begin{figure}[!htb]
  \centering
  \includegraphics[width=0.75\textwidth]{assets/ronnieandplane.png}
  \caption{Intern Space Program}
  \label{fig:ronnieandplane}
\end{figure}

\subsection{Swag and Questions}

We will open the floor to any and all questions, offering select NASA “swag” for those brave enough to start the conversation. Furthermore, after the question and answer period ends, we will stick around to hear out individual questions and students specific interests, while also handing out the rest of the swag we have. 

This system of outreach is a great way to expose high school students who have ambition in pursuing STEM gain additional information about opportunities outside of academics. A presentation encouraging students to explore NASA and other team challenges in the future offers specific ways for these students to get involved and therefore help to usher in the next generation.

\subsection{Summary}

We hope this presentation informs up-and-coming students of the great opportunities NASA provides, as well encourages them to participate in meaningful projects in STEM. We hope to bring excitement to subjects commonly-viewed as hard or dry by students going through middle and high school. Finally, we want to \textit{show} them why STEM can be phenomenally interesting, exciting, and provide life changing opportunities.

\newpage

\newcommand{\adminendorsement}{

\section{Administrative Section}

\subsection{Institutional Letter of Endorsement}

}

\includepdf[scale=0.8, offset=0 -60, pages=1, pagecommand=\adminendorsement]{assets/administratorendorsement.pdf}

\iffalse

To whom it may concern:

I endorse this team to participate in the NASA S.U.I.T.S competition. After reading the requirements for the project, hearing their proposal, and reading over their paper I believe they have a strong comprehension of what needs to be done and have outlined clearly the steps that they need to take.

They are utilizing industry tools such as Kanban boards and Gantt which, in my experience, are proven to lead towards success. Moreover, I endorse this team due to their demeanor. They are confident but still humble which is perhaps the most important key for success. I truly hope you consider them.

Sincerely,

Min Sing, Ph.D

\fi

\newpage

\newcommand{\supervisorendorsement}{

\subsection{Statement of Supervising Faculty}

}

\includepdf[scale=0.8, offset=0 -50, pages=1, pagecommand=\supervisorendorsement]{assets/facultyendorsement.pdf}

\iffalse

It is my great pleasure to endorse this team for the NASA S.U.I.T.S competition. Over the past 5 to 6 weeks they have been very diligent; meeting with me on a regular basis and setting up a schedule to meet multiple times a week on their own. They have been exceptional with their time management skills, working on their proposal and developing multiple pieces for this project. The amount of headway they have made in this short period is a true testament to their work ethic.  

They have a clear, well thought out plan which they have been following and developing since they first brought this project to me. I truly believe that this team has accomplished a great amount and have demonstrated a clear understanding and strong grasp of the development process and how to innovate while maintaining practicality. I truly have enjoyed watching this project develop since its installation and am excited to see it continue.

Sincerely,

Mukund Iyengar

\fi

\newpage

\newcommand{\rightsofuse}{

\subsection{Statement of Rights of Use}

}

\includepdf[scale=1.0, offset=0 0, pages=1, pagecommand=\rightsofuse]{assets/rightsofuse.pdf}

\iffalse

{\setstretch{0.5}
As a team member for a proposal entitled “\maintitle” proposed by a team of higher education students from \schoolname, I will and hereby do grant the U.S. Government a royalty-free, nonexclusive and irrevocable license to use, reproduce, distribute (including distribution by transmission) to the public, perform publicly, prepare derivative works, and display publicly, any technical data contained in this proposal in whole or in part and in any manner for Federal purposes and to have or permit others to do so for Federal purposes only. Further, with respect to all computer software designated by NASA to be released as open source which is first produced or delivered under this proposal and subsequent collaboration, if selected, shall be delivered with unlimited and unrestricted rights so as to permit further distribution as open source. For purposes of defining the rights in such computer software, “computer software” shall include source codes, object codes, executables, ancillary files, and any and all documentation related to any computer program or similar set of instructions delivered in association with this collaboration. As a team member for a proposal entitled “\maintitle” proposed by a team of higher education students from \schoolname, I will and hereby do grant the U.S. Government a nontransferable, irrevocable, paid-up license to practice or have practiced for or on behalf of the United States Government any invention described or made part of this proposal throughout the world.
}

\bigskip
\bigskip

\begin{tabular}{@{}p{.5in}p{3.5in}@{}}
Approved: & \hrulefill \\
& Schmidt, Joshua, 10/25/2019\\
\end{tabular}

\bigskip
\bigskip

\begin{tabular}{@{}p{.5in}p{3.5in}@{}}
Approved: & \hrulefill \\
& Regan, Dylan, 10/25/2019\\
\end{tabular}

\bigskip
\bigskip

\begin{tabular}{@{}p{.5in}p{3.5in}@{}}
Approved: & \hrulefill \\
& Weeks, Emily, 10/25/2019\\
\end{tabular}

\bigskip
\bigskip

\begin{tabular}{@{}p{.5in}p{3.5in}@{}}
Approved: & \hrulefill \\
& Cinquegrana, John, 10/25/2019\\
\end{tabular}

\bigskip
\bigskip

\begin{tabular}{@{}p{.5in}p{3.5in}@{}}
Approved: & \hrulefill \\
& Iyengar, Mukund, 10/25/2019\\
\end{tabular}

\fi

\newpage

\subsection{Funding and Budget Statement}

\bigskip

\hspace{1.275cm} Number of People Accommodating: 5

\bigskip

Estimated Number of Days: 7

\bigskip

Estimated Number of Nights: 6

\bigskip
\bigskip

\begin{tabular}{||l l c||} 
  \hline
  Category & Unit Price Estimate & Category Total \\ [0.5ex]
  \hline\hline
  Parts and hardware & \$600 total & \$600 \\ 
  \hline
  Flights & \$250 per Person & \$1250 \\ 
  \hline
  Hotel/Housing & \$90 per Night & \$540 \\
  \hline
  Food & \$30 per Person per Day & \$1050 \\
  \hline
  Ground Transportation & \$400 & \$400 \\
  \hline
  Miscellaneous & \$200 & \$200 \\
  \hline
   & Overall Total & \$4040 \\
  \hline
 \end{tabular}
 
\pagebreak

\subsection{Proposal Scoring Method}

\begin{figure}[!htb]
  \centering
  \includegraphics[width=0.75\textwidth]{assets/scoringsheet.png}
\end{figure}

\pagebreak

\newcommand{\raritansection}{

\section{Appendices}

\subsection{Raritan Confirmation letter}

This letter from Raritan High School in Hazlet, NJ, certifies the intent of our team to engage in an outreach program. As team member John Cinquegrana is an alumni of RHS he headed discussions with them. The RHS side of discussions was headed by Mr. Michael Miller (mmiller@hazlet.org) as described in the letter.

}

\includepdf[scale=0.68, offset=0 -70, pages=1, pagecommand=\raritansection]{assets/raritanacceptance.pdf}

\subsection{Pictures and Diagrams}

\begin{figure}[!htb]
  \centering
  \includegraphics[width=0.5\textwidth]{assets/slamdiagram.png}
  \caption{SLAM Localization}
  \label{fig:slamdiagram}
\end{figure}

\begin{figure}[!htb]
  \centering
  \includegraphics[width=0.75\textwidth]{assets/uwbdiagram.jpg}
  \caption{UWB localization}
  \label{fig:uwbdiagram}
\end{figure}

\begin{figure}[!htb]
  \centering
  \includegraphics[width=0.75\textwidth]{assets/hololenseyetracking.jpg}
  \caption{Hololens Eye Tracking}
  \label{fig:hololenseyetracking}
\end{figure}

\begin{figure}[!htb]
  \centering
  \includegraphics[width=0.85\textwidth]{assets/networkdiagram.jpg}
  \caption{Network Diagram}
  \label{fig:networkdiagram}
\end{figure}

\begin{figure}[!htb]
  \centering
  \includegraphics[width=0.75\textwidth]{assets/suitdiagnostics.png}
  \caption{Sample HUD of Suit Status}
  \label{fig:suitdiagnostics}
\end{figure}

\begin{figure}[!htb]
  \centering
  \includegraphics[width=0.75\textwidth]{assets/spacesuitdisplay.png}
  \caption{EVA Reference Notebook}
  \label{fig:spacesuitdisplay}
\end{figure}

\begin{figure}[!htb]
  \centering
  \includegraphics[width=0.75\textwidth]{assets/groundcontrolsuitmetrics.png}
  \caption{Suit Metrics Panel}
  \label{fig:groundcontrolsuitmetrics}
\end{figure}

\begin{figure}[!htb]
  \centering
  \includegraphics[width=0.75\textwidth]{assets/taskmanagement.png}
  \caption{Task Entry and View Panel}
  \label{fig:taskmanagment}
\end{figure}

\begin{figure}[!htb]
  \centering
  \includegraphics[width=0.85\textwidth]{assets/tasklistdropdown.jpg}
  \caption{Task List Dropdown}
  \label{fig:tasklistdropdown}
\end{figure}

\pagebreak

\begin{figure}[!htb]
  \centering
  \includegraphics[width=0.75\textwidth]{assets/waybackhome.jpg}
  \caption{“Way Back Home” Direction Indicator}
  \label{fig:wayhome}
\end{figure}

\end{document}
